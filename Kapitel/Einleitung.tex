%% Einleitung.tex
%%

%% ==============================
\chapter{Einleitung}
\label{ch:Einleitung}
%% ==============================



\begin{quotation}
\itshape
I have done something very bad today by proposing a particle that can not be detected. That's something no theorist should ever do. \newline
\normalfont
- Wolfgang Pauli 
\end{quotation} 


Im Jahre 1930 postulierte Wolfgang Pauli das Neutrino, wodurch die bis dahin ungekl�rte Form des kontinuierlichen Spektrums bei Beta-Zerf�llen erkl�rt werden konnte.
Das elektrisch neutrale Neutrino muss eine sehr kleiner Masse haben und war somit mit den damaligen Experimenten nicht Nachweisbar.
Erst 26 Jahre sp�ter gelang es beim Cowan-Reines-Neutrinoexperiment \cite{COWANREINES} Neutrinos nachzuweisen.
Und auch heute noch sind Neutrinos und ihre Eigenschaften fester Bestandteil vieler Experimente und von gro�em Interesse f�r die Physik.
Vor kurzem Schokierte OPERA \cite{OPERA} die Physik-Welt mit dem Ergebniss, dass Neutrinos 60ns schneller als Licht w�ren. 
RENO ver�ffentlicht vermutlich im Juni 2012 ihre Ergebnisse f�r den Mischungswinkel $\theta_{13}$ und KATRIN m�chte im Jahre 2015 eine neue Obergrenze f�r die Neutrinomasse gemessen haben. Okay, der letzte Satz ist schei�e. Aber ich versuche mal im Flow zu bleiben.

 
test \cite{RGr}







%\cite{OPERA}
lalala